%
\chapter{Commutation de paquets}
%
    \section{Un commutateur de paquets}
%
        \subsection{Énoncé}
%
            \paragraph{}
Nous cherchons à simuler un lien de sortie d'un commutateur de paquets.
            %    \begin{figure}
            %        \begin{center}
                        %\includegraphics[scale=0.5]{RSC/2.1.png}
            %        \end{center}
            %            \caption{ Schéma de fonctionnement d'un commutateur de paquets }
            %            \label{ Schéma de fonctionnement d'un commutateur de paquets }
            %    \end{figure}
%
            \paragraph{}
L'arrivée des paquets est supposée suivre une loi exponentielle de paramètre $\lambda$.
Nous positionnons une file en sortie du commutateur pour stocker les différents paquets.
Les paquets ont une longueur exponentielle-ment distribuée de paramètre $\frac{1}{\nu} = 10 000 bits$.
Le lien de sortie a un débit de 10 Mbit/s.
%
        \subsection{Calcul analytique du temps moyen de service $\frac{1}{\mu}$}
\[  \text{Temps moyen de service} = \frac{1}{\mu} \]
\[ \iff \frac{1}{\nu} \frac{1}{D} = 10^{4} \frac{1}{10^{7}} = \frac{1}{10^{3}} = 10^{-3} \ \text{seconde} \]
%
        \subsection{Déterminer le nombre moyen de paquets dans la file et le temps moyen de réponse en fonction du taux d'arrivée pour différentes durées de simulation}
\[  \lambda = \rho * \mu \]
\[  \text{Charge de trafic} \ \rho = \frac{\lambda}{\mu} \]
\[  \text{Nombre moyen de client} \ \bar{N} = \frac{\rho}{(1 - \rho)} \]
\[  \text{Temps moyen de reponse} \ \bar{W} = \frac{1}{(\mu - \lambda)} \]
\[  \bar{N} = \lambda * \bar{W} \]
\begin{center}
    \begin{tabular}{ | c | c| c | c | }
        \hline
            $\rho$ & 0.1 & 0.5 & 0.9 \\
        \hline
            $\lambda$ & $10^{2}$ & $5*10^{2}$ & $9*10^{2}$ \\
        \hline
            Nombre moyen de client & 0.11111 & 1 & 9 \\
        \hline
            Temps moyen de réponse & 0.00111 & 0.002 & 0.01 \\
        \hline
    \end{tabular}
\end{center}
%
        \subsection{Comparaison du résultat de la simulation avec la théorie}
            \paragraph{}
Blablabla 1
    % TODO >>>
    % SIMULATION
    % ici graphique
    % TODO <<<
%
        \subsection{Cas où les paquets ont une longueur constante (10000 bits)}
%
            \subsubsection{Calculer analytiquement le temps moyen de service $\frac{1}{\mu}$}
%
                \paragraph{}
On obtient la même chose que précédemment :
\[  \text{Temps moyen de service} = \frac{1}{\mu} \]
\[ \iff \frac{1}{\nu} * \frac{1}{D} = 10^{4} * \frac{1}{10^{7}} = \frac{1}{10^{3}} = 10^{-3} \ \text{seconde} \]
%
            \subsubsection{Résultats en fonction du taux d'arrivée pour différentes durées de simulation}
%
% voir pour trouver des formules, la progression doit être linéaire en fonction de $\rho$.
%
                \paragraph{Temps moyen de réponse}
Blablabla 3
%
                \paragraph{Nombre moyen de paquets dans la file d'attente}
Blablabla 4
%
            \subsubsection{Analyse et comparaison des résultats}
%
                \paragraph{}
Blablabla 5
    % TODO >>>
    % SIMULATION
    % ici graphique
    % TODO <<<
%
    \clearpage
%
