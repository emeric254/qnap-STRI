
\documentclass[a4paper,11pt]{report}
\usepackage[utf8]{inputenc}
\usepackage{color,amsmath,xcolor,listings,graphicx}
\usepackage[francais]{babel}

% paramétrage pour les zones de code perl
\lstset{
    language=Perl, commentstyle=\textit, frame=shadowbox,
    rulesepcolor=\color{gray}, basicstyle=\ttfamily\small, columns=flexible,
    tabsize=3, extendedchars=true, showspaces=false,
    showstringspaces=false, numbers=left, numberstyle=\tiny,
    breaklines=true, breakautoindent=true, captionpos=b, morecomment=[l]{//}
}

%language=Octave %-> choose the language of the code
%basicstyle=\footnotesize %-> the size of the fonts used for the code
%numbers=left %-> where to put the line-numbers
%numberstyle=\footnotesize %-> size of the fonts used for the line-numbers
%stepnumber=2 -> the step between two line-numbers.
%numbersep=5pt -> how far the line-numbers are from the code
%backgroundcolor=\color{white} -> sets background color (needs package)
%showspaces=false -> show spaces adding particular underscores
%showstringspaces=false -> underline spaces within strings
%showtabs=false -> show tabs within strings through particular underscores
%frame=single -> adds a frame around the code
%tabsize=2 -> sets default tab-size to 2 spaces
%captionpos=b -> sets the caption-position to bottom
%breaklines=true -> sets automatic line breaking
%breakatwhitespace=false -> automatic breaks happen at whitespace
%morecomment=[l]{//} -> displays comments in italics (language dependent)


% infos du document
\title{QNAP}
\author{Boulic Guillaume, Émeric Tosi}
\date{\today}


%
\begin{document}

    \maketitle{} % Afficher la page de garde : Titre + Auteur(s) + Date de dernière compilation


%    \begin{figure} % on s'en fout de l'image moche, c'est juste pour test xD
%        \begin{center}
            %\includegraphics{network.png}
            %\includegraphics[height=128, width=128]{network.png}
            %\includegraphics[scale=0.5]{network.png}
%        \end{center}
%            \caption{ Laule } % ce qui apparait juste en dessous de l'image
%            \label{c'est styler !}
%    \end{figure}

    \setcounter{tocdepth}{1} % définir la profondeur de l'Index
    \renewcommand{\contentsname}{Sommaire} % renommer l'Index en Sommaire
    \tableofcontents{} % afficher l'Index
    \clearpage


% Differentes Parties / Chapitres / Autres fichiers à inclure :

\chapter*{Introduction}
\addcontentsline{toc}{chapter}{Introduction}
        \paragraph{}
Blablabla.
        \paragraph{}
Lorem ipsum dolor sit amet, consectetur adipisicing elit, sed doeiusmod tempor incididunt ut labore et dolore magna aliqua.
Ut enimad minim veniam, quis nostrud exercitation ullamco laboris nisi utaliquip ex ea commodo consequat.
Duis aute irure dolor inreprehenderit in voluptate velit esse cillum dolore eu fugiat nullapariatur.
Excepteur sint occaecat cupidatat non proident, sunt inculpa qui officia deserunt mollit anim id est laborum.
    \clearpage


%
%
\chapter{RTC : Réseau Téléphonique Commuté}
%
    \section{Analyse sur un lien}
        \label{seullien} % label pour reference
%
        \subsection{Énoncé}
%
            \paragraph{}
Considérons un lien d'un réseau à commutation de circuits permettant de véhiculer de la voix téléphonique.
%
            \begin{figure}[h]
                \centering
                \includegraphics[scale=0.5]{RSC/1-1.png}
                \caption{ Schéma du réseau à commutation de circuit étudié }
                \label{ Schema du reseau a commutation de circuit }
            \end{figure}
%
            \paragraph{}
Chacune des connexions nécessite un débit de 64 Kb/s bi-directionnels.
On peut multiplexer simultanément $C$ appels téléphoniques sur ce lien.
%
        \paragraph{}
Le nombre d'utilisateurs est suffisamment grand pour supposer que les arrivées des nouveaux appels suivent une loi de paramètre, les durées des appels sont supposées suivre une loi exponentielle de paramètre , (1\u = 3 min).
%
        \subsection{Probabilité de blocage d'appel en fonction de la charge $\rho$ et de la capacité $C$}
\[ P(\text{blocage}) = \frac{ \frac{ \rho^C }{ C! } }{ \sum\limits_{i=0}^C \frac{ \rho^i }{ i! } } \]
\begin{center}
    avec $\rho$ la charge en Erlang et $C$ la capacité.
\end{center}
    % TODO >>>
    % ici graphique calcul theorique avec k et rho qui changent
    % TODO <<<
\begin{figure}
    \centering
    \begin{gnuplot}[terminal=epslatex, terminaloptions=color dashed]

    set xlabel 'Charge (Erlangs)'
    set ylabel 'Taux de rejet'
    plot "qnap/partie1/calcul.p1.q1.data" u 2:1 w l t "taux de rejet"
    \end{gnuplot}
    \caption{Résultats de l'étude théorique.}
    \label{pic:p1q1}
\end{figure}
%
        \subsection{Simulation de la probabilité de blocage d'appel pour une charge comprise entre 10 et 70 Erlangs}
Blablabla 1
    % TODO >>>
    % SIMULATION
    % ici graphique
    % TODO <<<
\begin{figure}
    \centering
    \begin{gnuplot}[terminal=epslatex, terminaloptions=color dashed]

    set xlabel 'Charge (Erlangs)'
    set ylabel 'Taux de rejet'
    plot "qnap/partie1/p1.q2.data" u 2:1 w l t "taux de rejet"
    \end{gnuplot}
    \caption{Résultats de l'étude théorique.}
    \label{pic:p1q2}
\end{figure}
%
        \subsection{Variation de la capacité C pour une variation de la charge normalisée entre 0.5 et 1}
Blablabla 2
    % TODO >>>
    % SIMULATION
    % ici graphique
    % TODO <<<
%~ \begin{figure}
    %~ \centering
    %~ \begin{gnuplot}[terminal=epslatex, terminaloptions=color dashed]

    %~ set xlabel 'Charge (\%)'
    %~ set ylabel 'Taux de rejet'
    %~ plot "qnap/partie2/p2.q3.data" u 1:2 t "minT",\
        %~ "qnap/partie2/p2.q3.data" u 1:3 t "maxT",\
        %~ "qnap/partie2/p2.q3.data" u 1:4 w l t "moyenne temps traitement",\
        %~ "qnap/partie2/p2.q3.data" u 1:5 t "minA" axes x1y2,\
        %~ "qnap/partie2/p2.q3.data" u 1:6 t "maxA" axes x1y2,\
        %~ "qnap/partie2/p2.q3.data" u 1:7 w l t "moyenne nb paquets en attente" axes x1y2
    %~ \end{gnuplot}
    %~ \caption{This is a simple example using the epslatex-terminal.}%
    %~ \label{pic:epslatex}%
%~ \end{figure}
%
        \subsection{Comparaison des taux de blocage expérimental et théorique}
Blablabla 3
%
    \clearpage
%
%
%
    \section{Analyse sur un réseau de trois commutateurs}
%
        \subsection{Énoncé}
%
            \paragraph{}
Désormais, nous considérons le réseau composé des 3 nœuds suivant :
%
            \begin{figure}[h]
                \centering
                \includegraphics[scale=0.5]{RSC/1-2.png}
                \caption{ Schéma du réseau à 3 commutateurs de circuit étudié }
                \label{ Schema du reseau a 3 commutateurs de circuit }
            \end{figure}
%
            \paragraph{}
Les arrivées sont supposées Poissonniennes sur chacun des nœuds et le trafic se répartit équiprobablement entre les différents nœuds.
Les durées des appels sont supposées exponentielles de même paramètre que dans la première partie (1-a).
Nous ne considérons pas les appels locaux ni les appels qui n'aboutissent pas (absence).
%
        \subsection{Probabilités de blocage avec le chemin de débordement en cas de saturation du chemin direct}
Blablabla 4
%
        \subsection{Comparaison des résultats avec la partie \ref{seullien}}
        % (on choisira donc des charges de trafic et des capacités de liens équivalentes)
Blablabla 5
    % TODO >>>
    % ici graphique simulation ou juste calculs ??
    % TODO <<<
%~ \begin{figure}
    %~ \centering
    %~ \begin{gnuplot}[terminal=epslatex, terminaloptions=color dashed]

    %~ set xlabel 'Charge (\%)'
    %~ set ylabel 'Taux de rejet'
    %~ plot "qnap/partie2/p2.q3.data" u 1:2 t "minT",\
        %~ "qnap/partie2/p2.q3.data" u 1:3 t "maxT",\
        %~ "qnap/partie2/p2.q3.data" u 1:4 w l t "moyenne temps traitement",\
        %~ "qnap/partie2/p2.q3.data" u 1:5 t "minA" axes x1y2,\
        %~ "qnap/partie2/p2.q3.data" u 1:6 t "maxA" axes x1y2,\
        %~ "qnap/partie2/p2.q3.data" u 1:7 w l t "moyenne nb paquets en attente" axes x1y2
    %~ \end{gnuplot}
    %~ \caption{This is a simple example using the epslatex-terminal.}%
    %~ \label{pic:epslatex}%
%~ \end{figure}
%
        \subsection{Problèmes à très forte charge !}
            \paragraph{}
Une solution consiste à n'utiliser le chemin de débordement que lorsque celui-ci n'est pas très encombré (en dessous d'un certain seuil d'occupation sur chacun des liens).
Cela revient donc à laisser une marge M aux appels directs.
%
            \subsubsection{Commentaires}
Blablabla 6
%
            \subsubsection{Simulation en prenant une marge comprise entre 1 et 3}
Blablabla 7
    % TODO >>>
    % ici graphique simulation
    % TODO <<<
%~ \begin{figure}
    %~ \centering
    %~ \begin{gnuplot}[terminal=epslatex, terminaloptions=color dashed]

    %~ set xlabel 'Charge (\%)'
    %~ set ylabel 'Taux de rejet'
    %~ plot "qnap/partie2/p2.q3.data" u 1:2 t "minT",\
        %~ "qnap/partie2/p2.q3.data" u 1:3 t "maxT",\
        %~ "qnap/partie2/p2.q3.data" u 1:4 w l t "moyenne temps traitement",\
        %~ "qnap/partie2/p2.q3.data" u 1:5 t "minA" axes x1y2,\
        %~ "qnap/partie2/p2.q3.data" u 1:6 t "maxA" axes x1y2,\
        %~ "qnap/partie2/p2.q3.data" u 1:7 w l t "moyenne nb paquets en attente" axes x1y2
    %~ \end{gnuplot}
    %~ \caption{This is a simple example using the epslatex-terminal.}%
    %~ \label{pic:epslatex}%
%~ \end{figure}
%
    \clearpage
%

\clearpage


%
%
\chapter{Commutation de paquets}
%
    \section{Un commutateur de paquets}
%
        \subsection{Énoncé}
%
            \paragraph{}
Nous cherchons à simuler un lien de sortie d'un commutateur de paquets.
            %    \begin{figure}
            %        \begin{center}
                        %\includegraphics[scale=0.5]{RSC/2.1.png}
            %        \end{center}
            %            \caption{ Schéma de fonctionnement d'un commutateur de paquets }
            %            \label{ Schéma de fonctionnement d'un commutateur de paquets }
            %    \end{figure}
%
            \paragraph{}
L'arrivée des paquets est supposée suivre une loi exponentielle de paramètre $\lambda$.
Nous positionnons une file en sortie du commutateur pour stocker les différents paquets.
Les paquets ont une longueur exponentielle-ment distribuée de paramètre $\frac{1}{\nu} = 10 000 bits$.
Le lien de sortie a un débit de 10 Mbit/s.
%
        \subsection{Calculer analytiquement le temps moyen de service $\frac{1}{\mu}$}
\[  \text{Temps moyen de service} = \frac{1}{\mu} \]
\[ \iff \frac{1}{\nu} \frac{1}{D} = 10^{4} \frac{1}{10^{7}} = \frac{1}{10^{3}} = 10^{-3} \ \text{seconde} \]
%
        \subsection{Déterminer le nombre moyen de paquets dans la file et le temps moyen de réponse en fonction du taux d'arrivée pour différentes durées de simulation}
\[  \lambda = \rho \mu \]
\[  \text{Charge de trafic} \ \rho = \frac{\lambda}{\mu} \]
\[  \text{Nombre moyen de client} \ \bar{N} = \frac{\rho}{(1 - \rho)} \]
\[  \text{Temps moyen de reponse} \ \bar{W} = \frac{1}{(\mu - \lambda)} \]
\[  \bar{N} = \lambda \bar{W} \]
\begin{center}
    \begin{tabular}{ | c | c| c | c | }
        \hline
            $\rho$ & 0.1 & 0.5 & 0.9 \\
        \hline
            $\lambda$ & $10^{2}$ & $5 10^{2}$ & $9 10^{2}$ \\
        \hline
            Nombre moyen de client & 0.11111 & 1 & 9 \\
        \hline
            Temps moyen de réponse & 0.00111 & 0.002 & 0.01 \\
        \hline
    \end{tabular}
\end{center}
%
        \subsection{Comparer le résultat de la mesure avec le résultat théorique}
            \paragraph{}
Blablabla
    % TODO >>>
    % ici graphique simulation
    % TODO <<<
%
        \subsection{Reprendre les deux questions dans le cas où les paquets ont une longueur constante (10000 bits)}
%
            \subsubsection{Calculer analytiquement le temps moyen de service $\frac{1}{\mu}$}
%
                \paragraph{}
On obtient la même chose que précédemment :
\[  \text{Temps moyen de service} = \frac{1}{\mu} \]
\[ \iff \frac{1}{\nu} \frac{1}{D} = 10^{4} \frac{1}{10^{7}} = \frac{1}{10^{3}} = 10^{-3} \ \text{seconde} \]
%
            \subsubsection{Déterminer le nombre moyen de paquets dans la file et le temps moyen de réponse en fonction du taux d'arrivée pour différentes durées de simulation}
%
                \paragraph{}
Blablabla
voir pour trouver des formules, la progression doit être linéaire en fonction de $\rho$.
%
            \subsubsection{Analyse et conclusion}
%
                \paragraph{}
Blablabla
    % TODO >>>
    % ici graphique simulation
    % TODO <<<
%
    \clearpage
%

\clearpage


%
\chapter*{Conclusion}
\addcontentsline{toc}{chapter}{Conclusion}
        \paragraph{}
Too much bullshit here :P
    \clearpage


%
\chapter*{Résumé}
\addcontentsline{toc}{chapter}{Résumé}
        \paragraph{}
Blablabla ...
    \clearpage


%
\chapter*{Abstract}
\addcontentsline{toc}{chapter}{Abstract}
        \paragraph{}
Blblblbl ...
    \clearpage


%
%
\appendix{}
%
\chapter{Annexes}
%
%
    \section{Analyse sur un lien}
%
        \paragraph{}
            \label{seullien-script}
Script Python de calcul de la probabilité de blocage d'appel $P$ en fonction de la charge $\rho$ et de la capacité $C$.
\lstinputlisting{./qnap/partie1/calcul.p1.q1.py}
    \clearpage
%
        \paragraph{}
Simulation d'un lien d'un réseau à commutation de circuits.
\lstinputlisting{./qnap/partie1/p1q2}
    \clearpage
%
%
%
    \section{Analyse sur un réseau de trois commutateurs}
%
        \paragraph{}
Simulation d'un lien d'un réseau à 3 commutateurs de circuits.
\lstinputlisting{./qnap/partie1/p1q3-2}
    \clearpage
%
        \paragraph{}
Simulation d'un lien d'un réseau à 3 commutateurs de circuits avec marge pour le débordement.
\lstinputlisting{./qnap/partie1/p1q4}
    \clearpage
%
%
%
    \section{Un commutateur de paquets}
%
        \paragraph{}
Script Python de calcul pour un lien de sortie d'un commutateur de paquets.
\lstinputlisting{./qnap/partie2/calcul.p2.q1.py}
    \clearpage{}
%
        \paragraph{}
Simulation d'un lien de sortie d'un commutateur de paquets.
\lstinputlisting{./qnap/partie2/p2q3.qnp}
    \clearpage{}
%
        \paragraph{}
Script Python de calcul pour un lien de sortie d'un commutateur de paquets avec la taille des paquets constante.
\lstinputlisting{./qnap/partie2/calcul.p2.q2.py}
    \clearpage{}
%
        \paragraph{}
Simulation d'un lien de sortie d'un commutateur de paquets avec la taille des paquets constante.
\lstinputlisting{./qnap/partie2/p2CST}
    \clearpage
%

\clearpage


%
%\listoffigures % index des images du rapport
%\clearpage


% fin
\end{document}
